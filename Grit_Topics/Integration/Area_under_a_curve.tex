\documentclass{article}

\usepackage{fancyhdr}
\usepackage{amsmath}
\usepackage{pgfplots}
\usepackage{xlop}
\usepackage{graphicx}
\graphicspath{ {./images/} }
\pagestyle{fancy}
\fancyhf{}
\rhead{Grit}
\lhead{Area Under a Curve}
\rfoot{Page \thepage}

\begin{document}
 
\section{Introduction}

Say you wanted to find the area under the following curve. It's sort of a triangle, but not really, so you can't apply $\frac{bh}{2}$ to it. There must be a formula, call it $A(x)$, that will give us the area under this curve or even... any curve, right?

\begin{figure}[htbp!]
\centering
\includegraphics[scale=0.75]{oogadog}
\end{figure}
\noindent

\newpage
\section{Determining A(x)}

\noindent
Let's assume $A(x)$ actually exists. $A(x)$ gives the area under a curve, $f(x)$, from $x=0$ to any $x$ value. Of course, we don't want to limit ourselves to $[0,x]$. We want to be able to find the area under $f(x)$ between any interval, that is, $[x,x+h]$. \\

\noindent
Let's visualize what we've just described:

\begin{figure}[htbp!]
\centering
\includegraphics[scale=0.75]{ooga-1}
\end{figure}

\noindent
The area, $A$, under the curve in $[x,x+h]$ is

\begin{align}
A_{[x,x+h]} = A(x+h)-A(x)
\end{align}

\noindent
To verbalize the above expression, the area from $[0,x]$ subtracted from the area from $[0,x+h]$ yields the area in $[x,x+h]$. 

\newpage
\noindent
We don't know $A(x)$ yet, so the best we can do is try to find a formula achieving something similar to it. A simple shape that could fit under the curve and we know the area formula for is a rectangle. Let's draw a rectangle under the curve within our interval, $[x,x+h]$.

\begin{figure}[htbp!]
\centering
\includegraphics[scale=0.75]{oogawhy}
\end{figure}

\noindent
The area of the rectangle is

\begin{align}
A_{rectangle} &= (x+h - x)*f(x) \\
A_{rectangle} &= h*f(x)
\end{align}

\noindent
The area of the rectangle is approximately equal to the area under the curve at $[x,x+h]$. We can express this as 
\begin{align}
h*f(x) \approx A(x+h)-A(x)
\end{align}

\newpage

\noindent
The area of the rectangle is somewhat close to the actual area under the curve, but it's still not \textit{exactly} the same. We can draw another rectangle to represent the excess area our first rectangle doesn't cover.

\begin{figure}[htbp!]
\centering

\includegraphics[scale=0.75]{oogahuh}
\end{figure}

\noindent
The blue rectangular area and the green excess represent the exact area under the curve, but the excess is incalculable | it's a portion of another rectangle. Let's call this other dashed rectangle $DR$ and the blue rectangle $BR$. \\

\noindent
Let's call the excess area $E$. We can use $E$ to create the exact equation
\begin{align}
h*f(x) &= A(x+h)-A(x)+E
\end{align}

\noindent
Equation 5 says that the area of $BR$ is equal to the exact area under the curve minus the excess, $E$. \\

\noindent
$E$ will never be higher than the area of DR because $E$ is a portion of DR. In fact, why don't we try pushing $x$ and $x+h$ closer together (decrease $h$) and see what happens to $E$...

\begin{figure}[htbp!]
\centering
\includegraphics[scale=0.75]{oogagiga}
\end{figure}

\newpage

\noindent
We can push $x$ and $x+h$ even closer...

\begin{figure}[htbp!]
\centering
\includegraphics[scale=0.75]{oogatera}
\end{figure}

\noindent
Notice that as $h$ gets smaller, how much of DR is composed by the excess, $E$, increases. That is, as $h$ approaches $0$, $E$ approaches the area of the dashed rectangle. \\

\noindent
The fact that $E$ will always be smaller than the area of DR and that $E$ approaches that area (becomes nearly equal to it) lets us set up the inequality

\begin{align}
E \leq \lvert h[f(x)-f(x+h)] \rvert
\end{align} 

\noindent
where the RHS represents the area of the dashed rectangle. To clarify, $f(x)-f(x+h)$ is the height of DR and $h$ is the width. \\

\noindent
$E$ is quite arbitrary, so let's try expressing it by rearranging equation 5:

\begin{align}
h*f(x) &= A(x+h)-A(x)+E \\
\frac{E}{h} &= \frac{A(x+h)}{h} - \frac{A(x)}{h} - f(x)
\end{align}

\noindent
We'll keep $E$ in the form $\frac{E}{h}$ for the sake of the next few steps. \\

\noindent
We can rewrite our inequality in equation 6 as

\begin{align}
\frac{A(x+h)}{h} -  \frac{A(x)}{h} - f(x) &\leq \lvert\frac{h[f(x)-f(x+h)]}{h} \rvert \\
\frac{A(x+h)-A(x)}{h} - f(x) &\leq \lvert f(x)-f(x+h) \rvert
\end{align}

\noindent
We're going to have get a bit conceptual now. Again, $E$ is always less than or equal to the area of DR, or $\lvert h[f(x)-f(x+h)] \rvert$. As $h$ approaches 0, $f(x)=f(x+h)$, so $f(x)-f(x+h)=0$. \\

\noindent
Look at our inequality in equation 10. As $h$ approaches $0$, the RHS equals 0. So, as the RHS gets smaller, so must the LHS since the LHS must always be less than or equal to the RHS. \\

\noindent
For the LHS to equal 0, the following must be true as $h$ approaches 0:
\begin{align}
f(x) &= \frac{A(x+h)-A(x)}{h}
\end{align}

\noindent
Using the two ideas that the RHS equals 0 as $h$ approaches 0 and that equation 11 must be true as $h$ approaches 0, we can   rephrase equation 11 as a limit:

\begin{align}
\lim_{h \to 0} f(x) &= \lim_{h \to 0} \frac{A(x+h)-A(x)}{h}
\end{align}

\noindent
$h$ is irrelevant to the value of the LHS because $h$ is not a variable on the LHS. \\

\noindent
Thus,

\begin{align}
f(x) &= \lim_{h \to 0} \frac{A(x+h)-A(x)}{h} \\
f(x) &= \frac{d}{dx}A(x)
\end{align}

\noindent
And wouldn't you know it, $f(x)$ is the derivative of $A(x)$. That is, $A(x)$ is the integral of $f(x)$. Thus,

\begin{align}
A(x) = \int_{}^{}f(x)dx
\end{align}

\noindent
Let's not forget our goal! $A(x)$ is the formula for the area under the curve in $[0,x]$. We need to modify the equation to give us the area from $[x,x+h]$. Well, that's as simple as subtracting $A(x)$ from $A(x+h)$. That is, using the following the definite integral:

\begin{align}
A(x+h) - A(x) = \int_{x}^{x+h}f(x)dx
\end{align}

\noindent
For the sake of making our final statement pretty, let's call our target interval $[a,b]$. \\

\noindent
The area under a curve, $f(x)$, in any interval $[a,b]$ can be given by the definite integral:

\begin{align}
\int_{a}^{b}f(x)dx
\end{align}

\newpage
\section{Example}
The point of this document is to show why the definite integral happens to be the area under a curve, so an example simply showcasing the integral feels... unrelated. Nonetheless, let's flex. \\

\begin{figure}[htbp!]
\centering
\includegraphics[scale=0.75]{oogafinale}
\end{figure}

\noindent
The graph above represents the velocity of a person running over a time interval. We can find the displacement of this person, say, from 2 to 4 seconds using the definite integral. \\

\noindent
I know the area under the curve is the displacement, meters, because the product or "area relationship" between velocity and time results in $(m/s)(s) = m$. \\

\noindent
The integral of $x^{2}-2x-1$ is 
\begin{align}
\frac{x^{3}}{3}-x^{2}-x
\end{align}

\noindent
Notice how we don't care about $+C$ because the definite integral cancels out the $+C$. \\

\noindent
Completing the definite integral yields
\begin{align}
A_{[2,4]} &= \int_{2}^{4}(x^{2}-2x-1)dx \\
A_{[2,4]} &= (\frac{4^{3}}{3}-4^{2}-4) - (\frac{2^{3}}{3}-2^{2}-2) \\
A_{[2,4]} &= \frac{4}{3}-\frac{-10}{3} \\
A_{[2,4]} &= 14/3
\end{align}

\noindent
The displacement of the runner in the interval $[2,4]$ is $4.67$ meters.
\section{Conclusion}
\noindent
That was one hell of a ride just to realize that the definite integral was the answer all along. But hey, now we know what exactly that integral means and why we should care about integrals at all. \\

\noindent
Finding the area under a curve is extremely useful, especially in applied mathematics like physics, where you're graphing quantities against each other all the time. \\

\section{References}
\begin{verbatim}
http://www.math.uchicago.edu/~murphykate/MPPMathCamp2017/MPP_Integration.pdf
\end{verbatim}

\begin{verbatim}
https://www.math.ucdavis.edu/~hunter/m125b/ch1.pdf
\end{verbatim}

\begin{verbatim}
https://en.wikipedia.org/wiki/Fundamental_theorem_of_calculus
\end{verbatim}

\begin{verbatim}
my_calculus_teacher.exe
\end{verbatim}
\end{document}
\end{document}